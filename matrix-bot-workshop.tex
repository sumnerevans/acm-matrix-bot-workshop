\documentclass{acm}

\usepackage{fontawesome}
\usepackage{etoolbox}
\usepackage{textcomp}
\usepackage[nodisplayskipstretch]{setspace}
\usepackage{xspace}
\usepackage{verbatim}
\usepackage{multicol}
\usepackage{soul}
\usepackage{attrib}

\usepackage{amsmath,amssymb,amsthm}

\usepackage[linesnumbered,commentsnumbered,ruled,vlined]{algorithm2e}
\newcommand\mycommfont[1]{\footnotesize\ttfamily\textcolor{blue}{#1}}
\SetCommentSty{mycommfont}
\SetKwComment{tcc}{ \# }{}
\SetKwComment{tcp}{ \# }{}

\usepackage{siunitx}

\usepackage{tikz}
\usepackage{pgfplots}
\usetikzlibrary{decorations.pathreplacing,calc,arrows.meta,shapes,graphs}

\AtBeginEnvironment{minted}{\singlespacing\fontsize{10}{10}\selectfont}
\setmainfont{Open Sans Light}
\usefonttheme{serif}

\makeatletter
\patchcmd{\beamer@sectionintoc}{\vskip1.5em}{\vskip0.5em}{}{}
\makeatother

% Math stuffs
\newcommand{\Z}{\mathbb{Z}}
\newcommand{\R}{\mathbb{R}}
\newcommand{\N}{\mathbb{N}}
\newcommand{\lcm}{\text{lcm}}
\newcommand{\Inn}{\text{Inn}}
\newcommand{\Aut}{\text{Aut}}
\newcommand{\Ker}{\text{Ker}\ }
\newcommand{\la}{\langle}
\newcommand{\ra}{\rangle}

\newcommand{\yournewcommand}[2]{Something #1, and #2}

\newenvironment{question}[1]{\par\textbf{Question #1.}\par}{}

\newcommand{\pmidg}[1]{\parbox{\widthof{#1}}{#1}}
\newcommand{\splitslide}[4]{
    \noindent
    \begin{minipage}{#1 \textwidth - #2 }
        #3
    \end{minipage}%
    \hspace{ \dimexpr #2 * 2 \relax }%
    \begin{minipage}{\textwidth - #1 \textwidth - #2 }
        #4
    \end{minipage}
}

\newcommand{\frameoutput}[1]{\frame{\colorbox{white}{#1}}}

\newcommand{\tikzmark}[1]{%
\tikz[baseline=-0.55ex,overlay,remember picture] \node[inner sep=0pt,] (#1)
{\vphantom{T}};
}

\newcommand{\braced}[3]{%
    \begin{tikzpicture}[overlay,remember picture]
        \draw [thick,decorate,decoration={brace,raise=1ex,amplitude=4pt},blue] (#2.south west-|T1.south west) -- node[anchor=west,left,xshift=-1.8ex,text=olive]{#3} (#1.north west-|T1.south west);
    \end{tikzpicture}
}

\title{Matrix Bot Workshop}
\author{Sumner Evans and Robby Zampino}
\date{March 1, 2022}

\begin{document}

\begin{frame}{A bit about us}
    \begin{itemize}
        \item Sumner
            \begin{itemize}
                \item Graduated from Mines in 2019 with a master's in CS.
                \item Work at Beeper, a company that is building a Matrix-based
                    chat app.
                \item Teaching \textit{CSCI 406 Algorithms} and previously
                    \textit{CSCI 400} and \textit{CSCI 564}.
            \end{itemize}

        \item Robby
            \begin{itemize}
                \item Graduated from Mines in 2019 with a bachelor's in EE and
                    minor in CS.
              \item Founder and admin of ohea.xyz, an internet collective
                    (ask me if you want an ohea.xyz Matrix account).
            \end{itemize}
    \end{itemize}

    We became interested in Matrix when we were looking for an open source chat
    platform for ACM!
\end{frame}

\begin{frame}{Overview}
    \setbeamertemplate{section in toc}[sections numbered]
    \tableofcontents[hideallsubsections]
\end{frame}

\section{Introducing Matrix}

\begin{frame}{What is Matrix?}
    Matrix is an \textbf{open} specification for \textbf{encrypted},
    \textbf{decentralized} communication.
\end{frame}

\begin{frame}{Matrix is an \textit{open specification}}
    Open specifications and standards are all around you. They just make
    sense\texttrademark.

    Examples:\pause
    \begin{itemize}
        \item Power plugs
        \item USB
        \item Wi-Fi
        \item Every crypto algorithm that's any good
    \end{itemize}
    \pause

    Open protocols allow for \textit{open development} and  \textit{clean-room
    implementations}, they \textit{encourage competition}, and are
    \textit{externally auditable}.
\end{frame}

\begin{frame}{Matrix is \textit{encrypted} by default*}
    Matrix has encryption built-in.

    The core of the encryption is \textbf{Olm}, which is a clone of the Signal
    double-ratchet protocol.
    \begin{itemize}
        \item If a single key is compromised, the attacker cannot see past
            messages. This is called \textbf{forward secrecy}.
        \item Key exchanges happen often, and if an attacker misses a single key
            exchange, they are once again locked out. This is called
            \textbf{break-in recovery}.
    \end{itemize}
    \pause

    You end up with 1:1 Olm ratchets between all participants in the room.
    \pause

    Those ratchets are used to share the key data for the group ratchet (called
    Megolm) which is used to encrypt messages.
\end{frame}

\begin{frame}{Matrix is \textit{decentralized}}
    The Matrix architecture is actually a \textit{federated} architecture.

    Individual devices communicate to a \textit{homeserver} which anyone can
    host.

    The homeserver communicates with other homeservers in the federation.
    \pause

    Think of it like email. You can email somebody using Outlook from Gmail.*
    \pause

    Every server in the federation gets a copy of a room, so no one entity
    controls the network.
    \pause

    This also means that the network is resilient to censorship, individual
    server outages, or even wider internet outages.
\end{frame}

\begin{frame}{Matrix allows for \textit{bridges} and \textit{bots}}
    Bridges bring external chat networks into Matrix. (Sumner gave a talk about
    these earlier in the year.)
    \pause

    Bots allow for automated interactions and notifications. This is the focus
    of this workshop.
\end{frame}

\section{A brief overview of how Matrix works}

\begin{frame}{Two APIs}
    The \textbf{Client-Server API} specifies how clients communicate with their
    homeserver.

    This is the one we care about. \pause

    The \textbf{Server-Server API} or \textbf{Federation API} specifies how
    servers communicate with other servers to ensure that everyone has the same
    room state.
\end{frame}

\begin{frame}{Everything is an \textbf{event}}
    Everything* in Matrix is an event or a room. There are two main event types:
    \textbf{message events} and \textbf{state events}.
    \pause

    \textbf{Message events:} \\
    These describe transient `once-off' activities in a room such as an instant
    message, VoIP call setup, file transfer, etc. They generally describe
    communication activity.
    \pause

    \textbf{State events:} \\
    These describe updates to a given piece of persistent information (`state')
    related to a room, such as the room's name, topic, membership, participating
    servers, etc. State is modelled as a lookup table of key/value pairs per
    room, with each key being a tuple of \texttt{state\_key} and \texttt{event
    type}. Each state event updates the value of a given key.

    {
        \tiny
        See \url{https://matrix.org/docs/spec/\#room-structure}
    }
\end{frame}

\section{Writing a Matrix Bot}

\begin{frame}{Maubot}
    We will be showing you how to use Maubot, a Matrix bot framework by Tulir
    (one of Sumner's coworkers).

    Maubot is a Python bot framework. It provides nice utilities on top of the
    \texttt{mautrix-python} library. \pause

    \textbf{Admin Panel:} We have set up a maubot instance here:

    \begin{center}
        \Large
        \url{https://argon.ohea.xyz/_matrix/maubot/}
    \end{center}

    Username: \texttt{demo} \\
    Password: \texttt{matrixiscool}
\end{frame}

\begin{frame}[fragile]{General bot-writing workflow}
    \textbf{Documentation:}
    \url{https://docs.mau.fi/maubot/dev/getting-started.html}

    \bigskip

    \textbf{Install maubot} on your local machine:
    \begin{minted}{shell}
    pip3 install --user maubot
    \end{minted}

    \textbf{Writing your bot:}
    \begin{itemize}
        \item Run \texttt{mbc init} for interactive plugin creation.
        \item Run \texttt{mbc login} to connect to the maubot instance.
        \item After you write your code, upload it to the server with: \\
            \texttt{mbc build --upload}
        \item Create a user for a client: \\
            \texttt{mbc auth --register --homeserver argon.ohea.xyz --server argon}
        \item Create a client and instance on the Maubot Manager.
    \end{itemize}
\end{frame}

\begin{frame}[standout]
    \Huge
    Example: Echobot
\end{frame}

\begin{frame}[standout]
    \Huge
    Example: Firefighter bot
\end{frame}

\begin{frame}{Bot Ideas}
    \textbf{Now it's your turn!} We recommend you start with something small (an
    existing bot for example), then build up from there.

    \begin{itemize}
        \item Try writing a cooler echobot or a better reaction bot.
        \item A bot that sends back ASCII-art of the given word or phrase.
        \item A bot that applies a filter to images sent in a room.
        \item A 20 questions bot.
        \item A dice roll bot
        \item Welcome bot that says hi to new users.
        \item Upvote/downvote tracker bot that tracks the number of thumbs up
            and thumbs down emojis that each user gets.
        \item Any other ideas?
    \end{itemize}
\end{frame}

\end{document}
% Local Variables:
% TeX-command-extra-options: "-shell-escape"
% End:
